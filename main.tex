% This example is meant to be compiled with lualatex or xelatex
% The theme itself also supports pdflatex
\PassOptionsToPackage{unicode}{hyperref}
\documentclass[aspectratio=1610, 9pt]{beamer}

% Load packages you need here
\usepackage{polyglossia}
\setmainlanguage{german}

\usepackage{csquotes}


\usepackage{amsmath}
\usepackage{amssymb}
\usepackage{mathtools}

\usepackage{hyperref}
\usepackage{bookmark}

\usepackage{xfrac}
\usepackage[shortcuts]{extdash}

\usepackage[
  backend=biber,
]{biblatex}
% Quellendatenbank
\addbibresource{lit.bib}

\usepackage[
  locale=DE,                   % deutsche Einstellungen
  separate-uncertainty=true,   % immer Fehler mit \pm
  per-mode=symbol-or-fraction, % / in inline math, fraction in display math
]{siunitx}

\usepackage{svrsymbols}
\usepackage{cancel}

% \AtBeginSection[]{
  % \frame{
    % \frametitle{Inhaltsverzeichnis}
      % \tableofcontents[currentsection]
  % }
% }

% load the theme after all packages

\usetheme[
  showtotalframes, % show total number of frames in the footline
]{tudo}

% Put settings here, like
\unimathsetup{
  math-style=ISO,
  bold-style=ISO,
  nabla=upright,
  partial=upright,
  mathrm=sym,
}

\title{W-Massenmessung}
\author[Julia ~Sobolewski]{Julia Sobolewski}
\institute[Fakultät Physik]{Fakultät Physik}
\titlegraphic{\includegraphics[width=0.7\textwidth]{images/cdf.jpg}}


\begin{document}

\maketitle

\begin{frame}[allowframebreaks]{Inhaltsverzeichnis}
  \tableofcontents
\end{frame}

\section{Einleitung}

\subsection{Was sind W-Bosonen?}

\begin{frame}{Was sind W-Bosonen?}
  \begin{columns}
    \begin{column}{0.4\textwidth}
      \begin{figure}
        \includegraphics[width=\textwidth]{images/standard_model.png}
        \caption{Standardmodell der Teilchenphysik \cite{standard_model}}
        % \label{}
      \end{figure}
    \end{column}
    \begin{column}{0.4\textwidth}
      \begin{itemize}
        \item Eichboson \rightarrow Elementarteilchen
        \item vermittelt in der elektroschwachen Theorie die geladenen Ströme
        \item Ladung: $q = \pm e$
        \item Spin: $s = 1$
        \item mittlere Lebensdauer: $\SI{3e-25}{\s}$
        \item Masse: $m_W = \SI{80.379(12)}{\GeV}$
      \end{itemize}
    \end{column}
  \end{columns}
\end{frame}

\subsection{Entdeckung des W-Bosons}
\begin{frame}{Entdeckung des W-Bosons}
  \begin{columns}
    \begin{column}{0.4\textwidth}
      \begin{figure}
        \includegraphics[width=\textwidth]{images/feynman.png}
        \caption{Feynman-Diagramm niedrigster Ordnung zur Erzeugung von W-Bosonen \cite{skript}}
        \label{fig:feynman}
      \end{figure}
    \end{column}
    \begin{column}{0.4\textwidth}
      \begin{itemize}
        \item 1983 am Super Proton Synchrotron (SPS)
        \item Im naiven Partonmodell entsteht das W-Boson bei Kollision eines Valenzquarks des Protons $(\quarku, \quarkd)$ mit einem Valenzantiquark des Antiprotons $(\antiquarku, \antiquarkd)$
        % \item Valenzquark und -antiquark tragen je einen Impulsanteil von $x_{1,2} \approx \SI{0,2}{}$ des (Anti-)Protons
        % \item Benötigte Parton-Parton-Schwerpunktsenergie $\sqrt{\hat{s}} = \SI{80}{\GeV}$ \rightarrow benötigte $\proton \antiproton$-Schwerpunktsenergie $\sqrt{s} = \sqrt{\frac{\hat{s}}{x_1 x_2}} \approx \SI{400}{\GeV}$
      \end{itemize}
    \end{column}
  \end{columns}
\end{frame}

\begin{frame}
  \begin{itemize}
    \item Valenzquark und -antiquark tragen je einen Impulsanteil von $x_{1,2} \approx \SI{0,2}{}$ des (Anti-)Protons
    \item Um ein W-Boson zu erzeugen, wird eine Parton-Parton-Schwerpunktsenergie von $\sqrt{\hat{s}} = \SI{80}{\GeV}$ und somit eine $p \bar{p}$-Schwerpunktsenergie von $\sqrt{s} = \sqrt{\frac{\hat{s}}{x_1 x_2}} \approx \SI{400}{\GeV}$ benötigt
    \item Solche Schwerpunktsenergien waren zuerst am SPS vorhanden ($\sqrt{s} = \SI{540}{\GeV}$)
  \end{itemize}
\end{frame}

\subsection{Motivation}
\begin{frame}{Motivation}
  \begin{columns}
    \begin{column}{0.5\textwidth}
      \begin{figure}
        \includegraphics[width=\textwidth]{images/gfitter.png}
        \caption{Fit der schwachen Wechselwirkung \cite{gfitter}}
        \label{fig:gfitter}
      \end{figure}
    \end{column}
    \begin{column}{0.4\textwidth}
      \begin{itemize}
        \item W- und Z-Masse bestimmen zusammen den schwachen Mischungswinkel
        \item durch genaue Kenntnis der W- und t-Masse lässt sich die Masse des Higgs-Bosons eingrenzen
      \end{itemize}
    \end{column}
  \end{columns}
\end{frame}

\subsection{Theoretische Grundlagen}
\begin{frame}{Theoretische Grundlagen}
   \begin{itemize}
     \item Im Gegensatz zum Z-Nachweis im Zerfall $Z \rightarrow \ell^+ \ell^-$ über die invariante Masse des Leptonpaares kann man hier die Vierervektoren der Zerfallsprodukte nicht vollstandig bestimmen
     \item longitudinaler Impuls $p_z$ des Schwerpunktsystems der Kollision ist, weil das System geboostet ist, nicht bekannt \\
     \begin{itemize}
       \item[$\rightarrow$] Lösung: Verwendung transversaler Größen
     \end{itemize}
   \end{itemize}
\end{frame}

\begin{frame}
  \begin{itemize}
    \item im Zerfall $W \rightarrow \ell \nu$ insbesondere die Transversalimpulse des Leptons $p^\ell_T$ und des Neutrinos $p^\nu_T$ von besonderem Interesse
    \item Der Transversalimpuls des Neutrinos kann nur indirekt über "fehlende transversale Energie"  $\cancel{\it{E}}_{T}$ bestimmt werden
    \item Wenn man annimmt, dass das Neutrino das einzige Teilchen ist, das undetektiert dem Detektor entkommt, kann man ̈uber die Erhaltung des Transversalimpulses $\sum \vec{p}_T$ die transversale Flugrichtung und Energie des Neutrinos bestimmen.
  \end{itemize}
\end{frame}

\begin{frame}
  \begin{itemize}
    \item eine weitere Observable ist die "transversale Masse" $m_T$
    \begin{equation*}
      m_T = 2 p^\ell_T p^\nu_T \left(1 - \cos{\left(\varphi^\ell-\varphi^\nu \right)} \right)
    \end{equation*}
    \begin{center}
      \small{$p^\nu_T = \cancel{\it{E}}_{T}$, $\varphi^\ell-\varphi^\nu \: \hat{=} \: \text{Öffnungswinkel zwischen den Transversalimpulsen des Leptons und des Neutrinos}$}
    \end{center}
    \item Im Ruhesystem des W-Bosons und unter Annahme einer verschwindenden Zerfallsbreite $\Gamma_W$ ist $p_T = \frac{m_W}{2} \sin{(\theta)}$, und somit
    \begin{equation*}
      m_T = m_W \sin{(\theta)}
    \end{equation*}
    \item Der differenzielle Wirkungsquerschnitt als Funktion von $m_T$ wird durch eine Variablentransformation $\mu = \frac{m_T}{m_W} = \sin{(\theta)}$ im Wirkungsquerschnitt gewonnen
    \begin{equation*}
      \frac{\symup{d}\sigma}{\symup{d}\mu} = \frac{\symup{d} \sigma}{\symup{d} \cos{\theta}} \left| \frac{\symup{d} \cos{(\theta)}}{\symup{d} \mu} \right|
    \end{equation*}
  \end{itemize}
\end{frame}

\begin{frame}
  \begin{columns}
    \begin{column}{0.5\textwidth}
      \begin{figure}
        \includegraphics[width=\textwidth]{images/jacobi-peak.png}
        \caption{Darstellung der Jacobi-Kante in idealisierter Form und im Experiment \cite{vorlesung}.}
        % \label{}
      \end{figure}
    \end{column}
    \begin{column}{0.4\textwidth}
      \begin{itemize}
        \item Man erhält für die Jacobi-Determinante dieser Variablentransformation
        \begin{equation*}
          \frac{\symup{d} \cos{(\theta)}}{\symup{d} \mu} = \frac{\symup{d}}{\symup{d}\mu} \sqrt{1-\mu^2} = - \frac{\mu}{\sqrt{1-\mu^2}}
        \end{equation*}
        \item Der differenzielle Wirkungsquerschnitt als Funktion von $m_T$ besitzt damit einen scharfen Knick bei $m_T = m_W$ , den man als "Jacobi-Kante" bezeichnet
      \end{itemize}
    \end{column}
  \end{columns}
\end{frame}

\begin{frame}
  \begin{itemize}
    \item Eine Jacobi-Kante tritt analog auch im differenziellen Wirkungsquerschnitt $\frac{\symup{d} \sigma}{\symup{d} p_T}$ bei einem Transversalimpuls von $p_T = \frac{m_w}{2}$ auf
    \item Im Experiment ist Jacobi-Kante verschmiert
    \begin{itemize}
      \item[\rightarrow] W-Boson wird i.A. nicht in Ruhe erzeugt
      \item[\rightarrow] W-Boson besitzt endliche Zerfallsbreite
      \item[\rightarrow] Detektorauflösung
      \item[\rightarrow] Unsicherheiten in der Rekonstruktion
    \end{itemize}
  \end{itemize}
\end{frame}

\section{Tevatron}

\subsection{Allgemeines}

\begin{frame}{Weltkarte}
  \begin{beamercolorbox}[center, wd=\textwidth]{titlegraphic}
    \includegraphics[width=\textwidth]{images/map.png}
  \end{beamercolorbox}%
\end{frame}

\begin{frame}{Tevatron}
    \begin{itemize}
      \item Betrieb durch das Fermilab (Batavia, Illinois)
      \item Proton-Antiproton-Beschleuniger
      \item der stärkste Beschleuniger nach dem LHC am CERN
      \item Schwerpunktsenergie: $\SI{1,96}{\TeV}$
      \item Umfang: $\SI{6}{\km}$
      \item Run I:
      \begin{itemize}
        \item $31.08.1992$ - $20.02.1996$
        \item integrierte Luminosität: $\SI{180}{\pico \barn^{-1}}$
      \end{itemize}
      \item Run II:
      \begin{itemize}
        \item $01.03.2001$ - $29.09.2011$
        \item integrierte Luminosität: $\SI{10}{\femto \barn^{-1}}$ pro Detektor
      \end{itemize}
      \item stillgelegt seit $29.09.2011$
    \end{itemize}
\end{frame}

\subsection{Beschleuniger-Kette}

\begin{frame}{Beschleuniger-Kette}
  \begin{figure}
    \includegraphics[width=0.54\textwidth]{images/accelerator_chain.jpg}
    \caption{Beschleuniger-Kette am Fermilab \cite{accelerator_chain}.}
    % \label{}
  \end{figure}
\end{frame}

\subsection{Detektoren}

\subsubsection{CDF}

\begin{frame}{CDF}
    \begin{figure}
      \includegraphics[width=0.55\textwidth]{images/CDF.png}
      \caption{Schematischer Aufbau des CDF-Detektors \cite{CDF_aufbau}}
      % \label{}
    \end{figure}
\end{frame}


\subsubsection{D0}

\begin{frame}{D0}
  \begin{figure}
    \includegraphics[width=0.65\textwidth]{images/d0.jpg}
    \caption{Schematischer Aufbau des D0-Detektors \cite{D0}}
    % \label{}
  \end{figure}
\end{frame}

\section{Messstrategie und Unsicherheiten am Beispiel von CDF Run II Daten}

\subsection{Event Generation}

\begin{frame}{Event Generation}
  \begin{itemize}
    \item wichtige Größen:
    \begin{itemize}
      \item[\rightarrow] Teilimpulse der Valenzquarks
      \item[\rightarrow] Transversalimpuls des W-Bosons $p_T$
    \end{itemize}
    \item Die Teilimpulse werden mithilfe von globalen Fits auf Hoch-Energie-Daten constrained und als Parton-Verteilungsfunktionen (PDF) dargestellt
    \item Die PDFs werden von unabhängigen Kollaborationen parametrisiert
    \item Die Unsicherheit bei Nutzung einer Parametrisierung der CTEQ (The Coordinated Theoretical-Experimental Project on QCD) liegt bei $\delta m_W(PDF) = \SI{15}{\MeV}$
  \end{itemize}


  % There are two important components of W boson production for measuring m W : the fractional
  % momenta of u and d quarks inside the proton, and the W p T . The u and d momenta determine
  % p W
  % z , which affects the m T distribution. The u and d fractional momenta are constrained from
  % global fits to high-energy data and embodied in parton distribution functions (PDFs) indepen-
  % dently parametrized by the CTEQ 6 and MRST 7 collaborations. Using a CTEQ prescription
  % for obtaining PDF uncertainties, the CDF collaboration has estimated δm W (P DF ) = 15 MeV.


\end{frame}

\begin{frame}
  \begin{itemize}
    \item Die Verteilung von $p_T$ wird durch einen Event Generator (resbos 8) simuliert
    \item Die benötigten Parameter werden überwiegend durch die $p_T$-Messung des Z-Bosons aus Run I constrained
    \item Die Unsicherheit der resbos-Parameter beträgt damit $\delta m_W (p^W_T) = \SI{13}{\MeV}$
  \end{itemize}

  % The W boson p T distribution is predicted by an event generator (resbos 8) that combines a
  % QCD next-to-leading-log calculation with three non-perturbative parameters fit from high energy
  % data. The dominant constraint on these parameters comes from the Z boson p T measurement in
  % Run 1. The generator and detector simulation predict the observed Run 2 Z boson p T spectrum
  % well (Fig. 1). The uncertainty on the resbos parameters results in δm W (p W
  % T ) = 13 MeV.

\end{frame}

\begin{frame}
  \begin{itemize}
    \item Im W-Zerfall hat das Abstrahlen eines Photons durch ein Lepton im Endzustand den größten Effekt auf die W-Massenmessung
    % \item Den größten Effekt auf die Rekonstruktion der W-Masse hat das Abstrahlen eines Photons durch ein Endzustandstlepton
    \item Durch das Abstrahlen verringert sich der Impuls des Leptons wodurch eine geringere W-Masse rekonstruiert wird
    \item Dies wird mithilfe von Simulationen korrigiert
    \item Nicht simuliert werden Photonemissionen im Anfangszustand, Interferenz und Terme höherer Ordnung
    \item[\rightarrow] $\num{20}$ ($\num{15}$) $\si{\MeV}$ für den $\mu$- ($e$-) Kanal

    % In the W decay, the most important effect for the W mass measurement is the radiation
    % of a γ from a final-state l ± . This radiation results in a reduced l ± momentum, potentially
    % affecting the inferred mass of the W boson. CDF bases its simulation of final-state radiation
    % on a QED next-to-leading order event generator (wgrad). Effects from initial-state radiation,
    % interference, and higher-order terms are not simulated, resulting in a 20 (15) MeV uncertainty
    % for the m W measurement in the μ (e) channel.
  \end{itemize}
\end{frame}

\subsection{Impulskalibration}

\begin{frame}{Impulskalibration}
  \begin{itemize}
    \item Der Impuls eines geladenen Teilchens wird durch seine Ablenkung im Tracker bestimmt
    \item Da $p \sim \frac{1}{r}$, wird der Impuls als eine Funktion des inversen Impulses des $J/\psi$ skaliert
    \item Zur Verbesserung der Auflösung wird die Position des Strahls bei Myon-Spuren von W- und Z-Zerfällen beim Track-Fit mitberücksichtigt
  \end{itemize}
  % A charged particle’s momentum is measured through its observed curvature in the tracker.
  % Since the momentum is inversely proportional to curvature, the momentum scale is measured
  % as a function of the mean inverse momentum of J/ψ muons and fit to a line. The line has zero
  % slope, verifying the applicability of the extracted scale to W boson decays.
  % To improve momentum resolution, muon tracks from W and Z decays use the beam position
  % as a point in the track fit. This constraint cannot be applied to J/ψ decays since they can be
  % separated from the beam line. Instead, Υ decays are used to verify that the beam constraint
  % produces no bias on the momentum calibration. A systematic uncertainty of 15 MeV accounts
  % for the observed difference in scale. Including the uncertainty due to tracker alignment, CDF
  % estimates an uncertainty of δm W (p T scale) = 25 MeV.
\end{frame}

\begin{frame}
  \begin{itemize}
    \item Dieser Contraint kann aber nicht auf $J/\psi$-Zerfälle angewendet werden, da diese auch außerhalb der Beamline auftreten können
    \begin{itemize}
      \item[\rightarrow] Verwendung von $\Upsilon$-Zerfällen, um sicherzustellen, dass der Beam-Constraint keinen Bias verursacht
    \end{itemize}
    \item Es ergibt sich eine Unsicherheit von $\SI{15}{\MeV}$ aufgrund der unterschiedlichen Skalierungen
    \item Zusammen mit der Unsicherheit aufgrund der Tracker-Anordnung ergibt sich eine Unsicherheit von $\delta m_W (p_T \: \text{scale}) = \SI{25}{\MeV}$
  \end{itemize}
\end{frame}

\subsection{Kalibration des em. Kalorimeters}

\begin{frame}{Kalibration des em. Kalorimeters}
  \begin{itemize}
    \item Das elektromagnetische Kalorimeter mithilfe der Elektronen-Tracks aus den W-Zerfällen kalibriert
    \item Die Kalorimeter-Energie wird so skaliert, dass $\frac{E}{p} = \num{1}$ gilt
    \item Um dies an eine energieabhängige Skala anzupassen, wird die $\frac{E}{p}$-Verteilung als Funktion der Elektron $E_T$ gefittet und mit einem Korrekturfaktor versehen
  \end{itemize}


  % Given the momentum calibration, electron tracks from W decays are used to calibrate the
  % electromagnetic calorimeter. The calorimeter energy is scaled such that the ratio of energy
  % to track momentum (E/p) is equal to 1. To correct for an energy-dependent scale, the E/p
  % distribution is fit as a function of electron E T and a correction applied.
  % The significant amount of material in the silicon detector inside the tracker affects the posi-
  % tion of the E/p peak. An uncertainty on the amount of material translates into an uncertainty
  % on the measured E scale. The fraction of events in the region 1.19 < E/p < 1.85 is a mea-
  % sure of the material. The extent to which this region is not well modelled results in a 55 MeV
  % uncertainty on the W mass. This uncertainty dominates the total δm W (E scale) of 70 MeV.
\end{frame}

\begin{frame}
  \begin{itemize}
    \item Die Menge an passivem Material im Silizium-Detektor innerhalb des Trackers beeinflusst die Position des $\frac{E}{p}$-Peaks
    \begin{itemize}
      \item[\rightarrow] Die Unsicherheit der Menge des Materials geht direkt in die Unsicherheit der Energieskala über
    \end{itemize}
    \item Dies kann nicht gut modelliert werden
    \begin{itemize}
      \item[\rightarrow] Unsicherheit aufgrund des Materials beträgt $\SI{55}{\MeV}$
    \end{itemize}
    \item Dies macht den Großteil der totalen Unsicherheit von $\delta m_W (E \: \text{scale}) = \SI{70}{\MeV}$ ausausmacht
  \end{itemize}
\end{frame}

\subsection{Bestimmung des hadronischen Recoils}

\begin{frame}{Bestimmung des hadronischen Recoils}
  \begin{columns}
    \begin{column}{0.4\textwidth}
      \begin{figure}
        \includegraphics[width=0.8\textwidth]{images/w-recoil.png}
        \caption{Recoil beim semileptonischen W-Zerfall \cite{recoil}.}
        % \label{}
      \end{figure}
    \end{column}
    \begin{column}{0.4\textwidth}
      \begin{itemize}
        \item Energie des hadronischen Recoils wird durch Summation über die komplette im Kalorimeter deponierte Energie bestimmt, ausgenommen die Energie des Leptons
        \item Die Detektorantwort auf die Hadronenergie wird als $R = \frac{u_\text{meas}}{u_\text{true}}$ definiert
        \begin{itemize}
          \item[\rightarrow] $u_\text{true} \: \hat{=} \: \text{Recoil-Energie des W-Bosons}$
        \end{itemize}
      \end{itemize}
    \end{column}
  \end{columns}


  % The hadronic recoil energy is measured by vectorially summing all the energy in the calorimeter,
  % excluding that contributed by the l. The detector response to the hadronic energy is defined as R = u meas /u true , where u true is the recoil energy of the W boson. The response is measured
% using Z → ll, since the l is measured more precisely than the hadronic energy.
% The hadronic energy resolution is modelled as having a component from the underlying event
% (independent of recoil) and a component from the recoiling hadrons. The model parameters are
% tuned using the resolution of Z → ll along the axis bisecting the leptons. This axis is the least
% susceptible to fluctuations in l energy. The recoil response and resolution uncertainty on the W
% mass is 50 MeV, of which 37 MeV is due to the model of the underlying energy resolution.
\end{frame}

\begin{frame}
  \begin{columns}
    \begin{column}{0.4\textwidth}
      \begin{figure}
        \includegraphics[width=\textwidth]{images/z-recoil.png}
        \caption{Recoil beim leptonischen Z-Zerfall \cite{recoil}.}
        % \label{}
      \end{figure}
    \end{column}
    \begin{column}{0.4\textwidth}
      \begin{itemize}
        \item Kalibriert wird $R$ mithilfe des $Z \rightarrow \ell \ell$-Zerfalls, da die Leptonenergie genauer bestimmt werden kann
        \item Es ergibt sich eine Unsicherheit von $\delta m_W (\text{recoil}) = \SI{50}{\MeV}$
      \end{itemize}
    \end{column}
  \end{columns}
\end{frame}

\subsection{Untergrund-Abschätzung}

\begin{frame}{Untergrund-Abschätzung}
  \begin{itemize}
    \item Häufiger Untergrund bei $W \rightarrow e \nu$ und $W \rightarrow \mu \nu$ Zerfällen
    \begin{itemize}
      \item $Z \rightarrow \ell \ell$, wobei ein $\ell$ nicht rekonstruiert wird
      \item $W \rightarrow \tau \nu \rightarrow \ell \nu \nu \nu$
      \item Dijet-Produktion, wobei ein hadronischer Jet als $\ell$ rekonstruiert wird
    \end{itemize}
    \item Im $\mu$-Sample kommt noch Untergrund aus der kosmischen Strahlung hinzu
  \end{itemize}
  % \begin{columns}
    % \begin{column}{0.4\textwidth}
%
    % \end{column}
    % \begin{column}{0.4\textwidth}
%
    % \end{column}
  % \end{columns}


  % The backgrounds common to the W → eν and W → μν samples are: Z → ll, where one l is not
  % reconstructed; W → τ ν → l3ν; and dijet production, with one hadronic jet misreconstructed as
  % an l. In addition, the μ sample includes background from cosmic rays and decays in flight. The
  % W and Z backgrounds are estimated using Monte Carlo. The dijet background estimation uses
  % events with significant energy surrounding the l to enhance hadronic background and obtain a
  % / T distribution is then fit using the W and jet distribu-
  % / T distribution. The data E
  % background E
  % tions as input. The cosmic ray background is determined using track hit timing information and
  % / T ) distribution to a combination
  % the decay-in-flight background estimated by fitting the ∆φ(l, E
  % of W and decay-in-flight distributions. These estimates result in δm W (background) = 20 MeV.
\end{frame}

\begin{frame}
  \begin{itemize}
    \item Die W- und Z-Untergründe werden mithilfe von Monte-Carlo-Simulationen abgeschätzt
    \item Die Abschätzung des Dijet-Untergrundes erhöht den hadronischen Untergrund, um eine $\cancel{\it{E}}_{T}$-Verteilung für den Untergrund zu gewinnen
    \begin{itemize}
      \item[\rightarrow] Die $\cancel{\it{E}}_{T}$-Verteilung wird mit den W- und Dijet-Verteilungen als Input gefittet
    \end{itemize}
    \item Der Untergrund aufgrund von kosmischer Strahlung wird mithilfe der Track-Hit Zeiten abgeschätzt
    \item Insgesamt ergibt sich eine Unsicherheit von $\delta m_W (\text{background}) = \SI{20}{\MeV}$
  \end{itemize}
\end{frame}

\subsection{Massenfit}

\begin{frame}{Massenfit}
  \begin{columns}
    \begin{column}{0.4\textwidth}
      \begin{figure}
        \includegraphics[width=\textwidth]{images/m-fit.png}
        \caption{Gefittete Masseverteilung des W-Bosons \cite{unsicherheiten}.}
        % \label{}
      \end{figure}
    \end{column}
    \begin{column}{0.4\textwidth}
      \begin{itemize}
        \item $m_T$-Verteilung wird für den $e$- und $\mu$-Kanal gefittet
        \item Die Daten sind dabei geblindet und es wird eine Kreuzvalidierung mit unabhängigen Datensätzen und Simulationen durchgeführt
      \end{itemize}
    \end{column}
  \end{columns}

  % Given the energy calibrations, recoil model, and background estimation, the m T distribution is
 % fit for the e and μ channels. The predicted line shape agrees with that of the data (Fig. 1).
 % The central value is blinded while CDF cross-checks the analysis with independent data sets
 % and simulation. Combining the two channels (Table 1) results in δm W = 76 MeV.
\end{frame}

\subsection{Unsicherheiten}

\begin{frame}{Unsicherheiten}
  \begin{figure}
    \includegraphics[width=\textwidth]{images/unsicherheiten.png}
    \caption{Unsicherheiten der W-Massenmessung in $\frac{\si{\MeV}}{c^2}$ bei der Nutzung von $\SI{0,2}{\femto \barn^{-1}}$ von CDF Run 2 Daten. In Klammern sind die Unsicherheiten aus CDF Run 1B \cite{unsicherheiten}.}
    % \label{}
  \end{figure}
  \begin{itemize}
    \item Die Komibation beider Kanäle ergibt eine Unsicherheit von $\delta m_W = \SI{76}{\MeV}$
  \end{itemize}
\end{frame}

\section{Zusammenfassung}

\begin{frame}{Zusammenfassung}
  \begin{columns}
    \begin{column}{0.4\textwidth}
      \begin{figure}
        \includegraphics[width=0.8\textwidth]{images/comparison2.png}
        \caption{Darstellung und Vergleich aller W-Massenmessungen \cite{comparison2}}
        % \label{}
      \end{figure}
    \end{column}
    \begin{column}{0.4\textwidth}
      \begin{itemize}
        \item World average: $m_W = \SI{80,379(12)}{\GeV}$
        \item Das Besondere an dieser Messung ist die hohe Präzision
        \item Analyse dauert deswegen sehr lange (ATLAS: 2011 bis 2018)
      \end{itemize}
    \end{column}
  \end{columns}
\end{frame}

\section{Fragen}

\begin{frame}
  \begin{beamercolorbox}[center, wd=\textwidth]{titlegraphic}
    \includegraphics[width=0.7\textwidth]{images/particle_zoo.jpg}
  \end{beamercolorbox}%
\end{frame}

% \begin{frame}{Transversale Größen}
 % \begin{columns}
   % \begin{column}{0.4\textwidth}
     % Text,    der    in    der    ersten    Spalte    steht
   % \end{column}
   % \begin{column}{0.4\textwidth}
     % Text,    der    in    der    zweiten    Spalte    steht
   % \end{column}
 % \end{columns}
% \end{frame}

% \section{Test}
%

\section{Literatur}
\begin{frame}[allowframebreaks]{Literatur}
  \printbibliography
\end{frame}

\nocite{comparison}
\nocite{tevatron}
\nocite{particle_zoo}
\end{document}
